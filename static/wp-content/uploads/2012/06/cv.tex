\documentclass[spanish]{simplecv}
\usepackage[utf8]{inputenc}
\setlength{\parskip}{\medskipamount}
\setlength{\parindent}{0pt}
\usepackage{booktabs}
\usepackage{textcomp}
\usepackage{url}
\usepackage{amsmath}
\usepackage{color}
\usepackage{graphicx}
\usepackage[absolute,overlay]{textpos}
\usepackage[margin=1in]{geometry}

\makeatletter

%% Because html converters don't know tabularnewline
\providecommand{\tabularnewline}{\\}

\usepackage[colorlinks,urlcolor=blue]{hyperref}
\hypersetup{pdfborder=0 0 1}
\hypersetup{pdfborderstyle={/S/U/W 0.5}}
\hypersetup{urlbordercolor=0 0 1}

\makeatother

\usepackage{babel}
\deactivatetilden

\begin{document}
\title{Currículum vitae}

\maketitle
\noindent \begin{center}
{\scriptsize Puede consultar la última versión de este currículum
vitae en \url{https://es.chuso.net/contacto/cv}}
\end{center}{\scriptsize}

\section{Información personal}

\begin{tabular}{ll}
\toprule 
\textbf{Nombre:} & Jesús Pérez Rey\tabularnewline
\midrule 
\textbf{Dirección:} & Oculto por privacidad, solicitar datos completos por e-mail\tabularnewline
\midrule 
\textbf{Teléfono:} & Oculto por privacidad, solicitar datos completos por e-mail\tabularnewline
\midrule 
\textbf{Skype:} & Oculto por privacidad, solicitar datos completos por e-mail\tabularnewline
\midrule 
\textbf{Correo electrónico:} & \textcolor{blue}{\underbar{\href{mailto:chescu@gmail.com}{chescu@gmail.com}}}\tabularnewline
\midrule 
\textbf{Fecha de nacimiento:} & 1983\tabularnewline
\midrule
\textbf{Permiso de conducción:} & B con coche propio y disponibilidad para desplazarse\tabularnewline
\bottomrule 
\end{tabular}

\section{Estudios}

\begin{itemize}
\item Hortonworks Data Platform Certified Administrator (HDPCA), número de licencia 3b2c86a3-5571-4415-9733-1398431ef674.\newline
\url{http://bcert.me/phhkiiys}
\item Finalizados estudios de Ingeniería Técnica en Informática
de Gestión por la Universidad de Vigo en la Escuela Superior de Ingeniería
Informática de Ourense con matrícula de honor en el proyecto de fin de carrera.\newline
Número de colegiado en el Colegio Profesional de Ingeniería Técnica en Informática de Galicia (CPETIG): 0201.
\item Linux Professional Institute “LPIC-1”, número de certificación: LPI000340144
\item SUSE Certified Linux Administrator (SUSE CLA), número de certificación: 10277456
\item ``Automatización de Gestión de Servidores con Puppe'', por Consultec y CONETIC (Confederación Española de Empresas de Tecnologías de la Información, Comunicaciones y Electrónica).
\item Arkeia Partner Technical Certification. Data protection technical training curriculum. Certificate number T2795502.
\item {}``Iniciación a los sistemas de autenticación basados en LDAP'',
curso de extensión universitaria de la Escuela Superior de Ingeniería
Informática de Ourense.
\end{itemize}

\section{Experiencia}

\begin{itemize}
\item Programación y mantenimiento desde septiembre de 2006 hasta diciembre de 2010 de la tienda en línea
\textcolor{blue}{\underbar{\href{http://www.losestores.com}{LosEstores.com}}},
programada en Linux, Apache, PHP y MySQL (LAMP).\newline
Se pueden pedir referencias
en \textcolor{blue}{\underbar{\href{mailto:info@losestores.com}{info@losestores.com}}}
o usando la información de contacto que figura en la página web.
\item Programador web en la empresa Vangote realizando la tarea de configuración y mantenimiento
de un servidor dedicado Red Hat Linux Enterprise Server que hospedaría la página web
Altachannel.com de uploading y streaming de vídeo. Programación en LAMP (Linux, Apache,
PHP y MySQL) y mantenimiento de dicha página web que proporciona servicios de compra/venta
de vídeos a sus clientes permitiendo clasificar los vídeos según distintos parámetros y gestionar el
nivel de acceso de los usuarios dependiendo de dichos parámetros y del tipo de usuario.
\item Desde marzo de 2010 a octubre de 2011 en la empresa TECOnSITE, S.L., mantenimiento
y desarrollo de páginas web en PHP, principalmente con el framework CodeIgniter, pero
también para osCommerce, Zen Cart, Prestashop, Joomla, etc. Así como mantenimiento y
seguridad de los servidores CentOS de la empresa, realizando tareas como configuración de los
servicios, instalación de servidores, análisis de seguridad, prevención y recuperación de ataques,
actualizaciones, \...\newline
Se pueden pedir referencias en \href{mailto:info@teconsite.com}{info@teconsite.com}.
\item Desde octubre de 2011 hasta febrero de 2016 en la empresa Ultreia Comunicaciones, S.L. en
administración e implantación de sistemas Linux de correo, web, servidores de ficheros Samba, OpenVPN,
firewalls, servicios de almacenamiento web (tipo Dropbox) con AjaXplorer, \...
Principalmente pero no exclusivamente, servidores Debian virtualizados con Citrix
Xen Server. Programación y mantenimiento de páginas web en PHP.\newline
Se pueden pedir referencias en \href{mailto:info@ultreia.es}{info@ultreia.es}.
\item Desde marzo de 2016 hasta la actualidad en Bluemetrix Ltd como consultor Hortonworks para HSBC en Londres realizando tareas de despliegue, mantenimiento de sus clusters Hadoop con Hortonworks Data Platform (HDP): HDFS, YARN, Ranger, ZooKeeper, Hive, Kafka, HBase…
\end{itemize}

\section{Otros trabajos}

\begin{itemize}
\item Colaboración desde abril de 2003 hasta la actualidad en la distribución
de Linux \href{http://www.gentoo.org}{Gentoo Linux}.\newline
Referencias: \url{http://chu.so/gentoo-contribs}.
\item Colaboración desde Abril de 2012 hasta febrero de 2016 con el software libre de almacenamiento online escrito en PHP \href{https://pyd.io/}{AjaXplorer/Pydio:}
Referencias:\newline
\url{https://github.com/pydio/pydio-core/commits/develop?author=chusopr}
\url{https://github.com/ultreia-es/pydio-core/commits/grandesficheros-6.0.2?author=chusopr}
\item Integración de KDE con SugarCRM desarrollando un plugin de SugarCRM para el framework de gestión de información personal de KDE \href{https://community.kde.org/KDE_PIM/Akonadi}{Akonadi} como proyecto de fin de carrera para la Universidad calificado con matrícula de honor.\newline
Código: \url{https://github.com/chusopr/akonadi-sugarcrm}\newline
Documentación: \url{https://chuso.net/akonadi-sugarcrm.pdf}
\item Otros trabajos y artículos pueden ser consultados en \href{https://chuso.net}{Chuso.net}.
\item Perfil de GitHub: \url{https://github.com/chusopr}.
\end{itemize}

\section{Idiomas}

\begin{itemize}
\item Español y gallego como lenguas madre.
\item Inglés escrito nivel medio/alto y hablado nivel medio.
\end{itemize}

\end{document}
